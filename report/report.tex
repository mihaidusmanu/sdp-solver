\documentclass{article}

\usepackage{amsmath}
\usepackage{amssymb}
\usepackage{graphicx}

\title{Semi-Definite Programming Solver}
\author{Mihai DUSMANU \& Remi JEZEQUEL}

\begin{document}
	
\maketitle

\section{Formulae}

\subsection{Primal and dual problem}

($P$)\begin{tabular}{l l}
	minimize$_{X}$ & $\mathbf{Tr}(C X)$ \\
	subject to & $\mathbf{Tr}(A_i X) = b_i, \forall i \in \{1, \dots, m\}$ \\
	& $X \succeq 0$  
\end{tabular}

$\mathcal{L}(X, Z, \mu) = \mathbf{Tr}(C X) - \mathbf{Tr}(X Z) + \sum_{1 \leq i \leq m} \mu_i (\mathbf{Tr}(A_i X) - b_i) \\
\implies \nabla_X \mathcal{L}(X, Z, \mu) = C^T - Z^T + \sum_{1 \leq i \leq m} \mu_i A_i^T \\
\implies g(Z, \mu) = \begin{cases}
- \sum_{1 \leq i \leq m} \mu_i * b_i = - \mu^T b & if\ Z = C + \sum_{1 \leq i \leq m} \mu_i A_i^T \\ 
- \infty & otherwise 
\end{cases}$

Thus, the dual is: \\
($D_0$)\begin{tabular}{l l}
	maximize$_{\mu, Z}$ & $- \mu^T b$ \\
	subject to & $Z = C + \sum_{1 \leq i \leq m} \mu_i A_i$ \\
	& $Z \succeq 0$
\end{tabular}

This problem can be simplified: \\
($D$)\begin{tabular}{l l}
	minimize$_{\mu}$ & $b^T \mu$ \\
	subject to & $C + \sum_{1 \leq i \leq m} \mu_i A_i \succeq 0$
\end{tabular}

\subsection{Log-barrier method}
We'll solve the dual problem by using the log-barrier method: \\
($LB$)\begin{tabular}{l l}
	minimize$_{\mu}$ & $t b^T \mu - \log(\det(C + \sum_{1 \leq i \leq m} \mu_i A_i))$
\end{tabular}

\begin{itemize}
	\item $f(\mu) = t b^T \mu - \log(\det(C + \sum_{1 \leq i \leq m} \mu_i A_i))$
	\item $F(\mu) = C + \sum_{1 \leq i \leq m} \mu_i A_i$
\end{itemize}

In order to use Newton's method, we need to compute: \begin{itemize}
	\item $\nabla f(\mu)$ \\
	$\nabla f(\mu) = t b - \begin{bmatrix}
	\mathbf{Tr}(F(\mu)^{-1} A_i)
	\end{bmatrix}_{1 \leq i \leq m}$
	\item $\nabla^2 f(\mu)x$ \\
	$\frac{d \nabla f(\mu)}{d \mu_j} = - \begin{bmatrix}
	\mathbf{Tr}([\frac{d \mathbf{Tr}(F(\mu)^{-1} A_i)}{dF(\mu)}]^T \frac{d F(\mu)}{d \mu_j})
	\end{bmatrix}_{1 \leq i \leq m}$ \\
	But $\frac{d F(\mu)}{d \mu_j} = A_j$, and $\frac{d \mathbf{Tr}(F(\mu)^{-1} A_i)}{dF(\mu)} = - F(\mu)^{-T} A_i^T F(\mu)^{-T} \\
	\implies \nabla^2 f(\mu) = \begin{bmatrix}
	\mathbf{Tr}(F(\mu)^{-1} A_i F(\mu)^{-1} A_j)
	\end{bmatrix}_{1 \leq i, j \leq m}$
\end{itemize}

\subsection{Primal solution from dual solution}

Let $\mu^*$ be an optimal solution of ($D$). $\frac{1}{t} F(\mu^*)$ is an optimal solution of ($P$).

\subsection{Rank 1 solution}

In this section we suppose that $m = 1$, i.e. the problem ($P$) can be rewritten as follows: \\
($P_{m = 1}$)\begin{tabular}{l l}
	minimize$_{X}$ & $\mathbf{Tr}(C X)$ \\
	subject to & $\mathbf{Tr}(A X) = b$ \\
	& $X \succeq 0$  
\end{tabular}

According to Appendix $B.3$ of "Convex Optimization" - Boyd \& Vandenberghe, for all $X \in \mathbf{S}^n_+$ such as $\mathbf{Tr}(C X) = c$ and $\mathbf{Tr}(A X) = b$, there exists an $x \in \mathbb{R}^n$ that satisfies $\mathbf{Tr}(C x x^T) = c$ and $\mathbf{Tr}(A x x^T) = b$. 

Thus if $X^*$ is an optimal solution of ($P_{m = 1}$) then we can find a vector $x^*$ such as the rank $1$ matrix $x^* (x^*)^T$ is an optimal solution as well. The proof given in the book is constructive so we implemented it directly.

\section{Random SDP problems}

\subsection{SDP problem with one equality constraint}
We pick $C$ a random $n \times n$ matrix. For $A$ and $b$, we pick values that guarantee that the problem will be bounded (e.g. $A = diag(1 : n), b = 1$).

\subsection{SDP problem with two (or more) equality constraints}
We pick $C$ a random $n \times n$ matrix. For $A_1$ and $b_1$, we pick values that guarantee that the problem will be bounded (e.g. $A = diag(1 : n), b = 1$). For the other constraints ($A_2, b_2, \dots$) we pick random values. This doesn't guarantee that the problem will be feasible.

\section{Comparison with "Convex.jl"}

We've noticed that the performance of our algorithm on randomly generated SDP problems is extremely variable (in some cases we get results that are close to the ones obtained with the module "Convex.jl", in some others the results are really different). We don't know if this comes from an error in our implementation or simply from numerical problems. 

\end{document}